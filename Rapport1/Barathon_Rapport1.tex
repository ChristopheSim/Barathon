\documentclass[a4paper,11pt]{article}
\usepackage[utf8]{inputenc}
\usepackage[french]{babel}

%opening
\title{Barathon: rapport intermédiaire}
\author{Bourguignon Maxime \\ Jacobi Jordan \\ Simon Christophe \\ Vanneste Jean}

\begin{document}

\maketitle


\section*{Introduction}
Dans ce document, nous allons aborder plusieurs aspects de notre projet appelé \og{} Barathon\fg{}. Ce rapport intermédiaire est composé de plusieurs parties qui sont:
\begin{itemize}
 \item Description du projet;
 \item Diagrammes réalisés;
 \item Conventions de codage;
 \item Critères de qualité.
\end{itemize}
Ce document n'est pas exhaustif mais est assez représentatif de l'idée que l'on se fait du \og{} Barathon \fg{}.

\section{Description du projet}
Blabla

\section{Diagrammes réalisés}
Dans cette section, nous allons aborder les deux diagrammes réalisés sur le projet.

\subsection{Diagramme des cas d'utilisation}
Blabla

\subsection{Diagramme de classes}
Blabla

\section{Conventions de codage}
BLabla

\section{Crières de qualité}
Blabla

\end{document}
