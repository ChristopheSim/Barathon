\documentclass[a4paper,11pt]{article}
\usepackage[utf8]{inputenc}
\usepackage{hyperref}
\usepackage[french]{babel}

%opening
\title{Barathon: rapport intermédiaire}
\author{Bourguignon Maxime \\ Jacobi Jordan \\ Simon Christophe \\ Vanneste Jean}

\begin{document}

\maketitle


\section*{Introduction}
Dans ce document, nous allons aborder plusieurs aspects de notre projet appelé \og{} Barathon\fg{}. Ce rapport intermédiaire est composé de plusieurs parties qui sont:
\begin{itemize}
 \item Description du projet;
 \item Diagrammes réalisés;
 \item Conventions de codage;
 \item Critères de qualité.
\end{itemize}
Ce document n'est pas exhaustif mais est assez représentatif de l'idée que l'on se fait du \og{} Barathon \fg{}.

\section{Description du projet}
Le projet Barathon doit permettre à un utilisateur de trouver facilement des bars à proximité et de se constituer un itinéraire. L'application est pensée pour tenir compte des préférences de l'utilisateur, de sa position géographique, des bars déjà visités, de son budget et éventuellement de son moyen de transport. Elle est pensée pour être déployée en tant qu'application mobile.

Après la création d'un profil de préférences, l'utilisateur pourra utiliser notre algorithme pour trouver le parcours qui correspond le mieux à ses critères.

\section{Diagrammes réalisés}
Dans cette section, nous allons aborder les deux diagrammes réalisés sur le projet.

\subsection{Diagramme des cas d'utilisation}
Blabla

\subsection{Diagramme de classes}
Blabla

\section{Conventions de codage}
Pour les conventions de codage, nous allons utiliser les conventions de codage standard de java.
Cela nous permettra de verifier facilement que notre code respecte celle-ci avec des applications tels que PMD ou Checkstyle.\\
les conventions de codage standard de java que nous suivont sont celles-ci:
\begin{itemize}
  \item Le nom des classes est en camelcase avec la première lettre en majuscule;
  \item Le nom des variables est en camelcase avec la première lettre en minuscule;
  \item Les variables d'une seule lettre sont utilisées localement;
  \item Le nom des constantes est en majuscule avec un underscore pour séparer les mots;
  \item Le nom des fichiers en minuscule avec des tirets pour séparer les noms;
  \item L'indentation est de 4 espaces;
  \item Les commentaires se trouvent au dessus des déclarations de classes.
\end{itemize}


Pour de plus ample informations, vous pouvez retrouver nos conventions de codage sur le site d'oracle:
\url{https://www.oracle.com/technetwork/java/codeconvtoc-136057.html}.

\section{Critères de qualité}
Comme critères de qualité, nous avons choisis:
\begin{itemize}
  \item Une densité de commentaire de 20 à30\%;
  \item Une couverture de code de 40\% au minimum pour la partie logique;
  \item Une rapidité d'exécution : exécution de l'algorithme de recherche d'itinéraire en moins de 30 secondes sur un téléphone mobile Android de dernière génération.
\end{itemize}


Comme c'est un travail qu'il faudra s'échanger, pour faciliter la relève,
nous voulons un programme qui est bien documenté, et qui est composé de fonctions fiables.\\
Pour notre projet, une application de recherche dans une database, nous voulons que celle-ci s'exécute vite.
Cette vitesse va dépendre de la requête et du hardware, mais nous voulons optimiser celle-ci le plus possible.

\end{document}
