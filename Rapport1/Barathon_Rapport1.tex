\documentclass[a4paper,11pt]{article}
\usepackage[utf8]{inputenc}
\usepackage{french}
\usepackage{hyperref}

%opening
\title{Barathon: rapport intermédiaire}
\author{Bourguignon Maxime \\ Jacobi Jordan \\ Simon Christophe \\ Vanneste Jean}

\begin{document}

\maketitle


\section*{Introduction}
Dans ce document, nous allons aborder plusieurs aspects de notre projet appelé \og{} Barathon\fg{}. Ce rapport intermédiaire est composé de plusieurs parties qui sont:
\begin{itemize}
 \item Description du projet;
 \item Diagrammes réalisés;
 \item Conventions de codage;
 \item Critères de qualité.
\end{itemize}
Ce document n'est pas exhaustif mais est assez représentatif de l'idée que l'on se fait du \og{} Barathon \fg{}.

\section{Description du projet}
Blabla

\section{Diagrammes réalisés}
Dans cette section, nous allons aborder les deux diagrammes réalisés sur le projet.

\subsection{Diagramme des cas d'utilisation}
Blabla

\subsection{Diagramme de classes}
Blabla

\section{Conventions de codage}
Pour les conventions de codage, nous allons utiliser les conventions de codage standard de java.
Cela nous permettra de verifier facilement que notre code respecte celle-ci avec des applications tels que PMD ou Checkstyle.\\
les conventions de codage standard de java que nous suivont sont celles-ci:
\begin{itemize}
  \item Le nom des classes est en camelcase avec la première lettre en majuscule;
  \item Le nom des variables est en camelcase avec la première lettre en minuscule;
  \item Les variables d'une seule lettre sont utilisées localement;
  \item Le nom des constantes est en majuscule avec un underscore pour séparer les mots;
  \item Le nom des fichiers en minuscule avec des tirets pour séparer les noms;
  \item L'indentation est 4 espaces;
  \item Les commentaires se trouvent au dessus des déclarations de classes.
\end{itemize}
Pour de plus ample informations, vous pouvez retrouver nos conventions de codage sur le site d'oracle:
\url{https://www.oracle.com/technetwork/java/codeconvtoc-136057.html}.

\section{Crières de qualité}
Blabla

\end{document}
